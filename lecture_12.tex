\section{Iдентифікація та автентифікація. Криптографічні протоколи аутентифікації}
\begin{flushright}
\emph{(Автор: Марина Соловйова. Частково редагувалось.)}
\par \emph{(Версія від 22 січня 2017 р.)}
\end{flushright}

\par\emph{Ідентифікація} - це надання суб'єкту (об'єкту) індивідуального кода (наприклад, номера), тобто ідентифікатора, який відрізняється від всіх інших, а також перевірка ідентифікатора(все це не секретні дані). \par
\emph{Автентифікація} - це перевірка достовірності суб'єкту по наданим ідентифікаторам та деякій додатковій інформації (наприклад, паролям, ключам) шляхом порівняння.
\parПри контакті користувача, наприклад, з комп'ютерною мережею здійснюються наступні процедури:

\begin{itemize}
\item ідентифікація
\item автентифікація
\item авторизація
\end{itemize}

\begin{center}
\textbf{Принципи аутентифікації}
\end{center}

\begin{enumerate}
\setcounter{enumi}{0}
\item По наданню деякої секретної інформації, відомої користувачеві (пароль, код).
\item По наданню деяких об'єктивних характеристик (мікросхема, магнітна стрічка).
\item По наданню індивідуальних біометричних характеристик (відбитків пальців, рисунку райдужної оболонки ока, тембру голосу).
\item По характеристикам роботи з апаратурою в реальному часі.
\item По відповідям в режимі діалогу.
\item За допомогою третьої сторони.
\end{enumerate}

\begin{center}
\textbf{Криптографічні протоколи аутентифікації}
\end{center}

\begin{enumerate}
\item Автентифікація за допомогою пароля  (на прикладі аутентифікації з комп'ютерною системою -- надалі будемо позначати як <<КС>>).
 \begin{enumerate}
    \item \textit{Парольна автентифікація з використанням односторонніх функцій}
    \parНехай h(x) -- одностороння хеш-функція, що володіє сильною стійкістю до колізій.
    \parКористувачі $A_{i}$ мають ідентифікатори $D_{i}$ та паролі $p_{i}$, i = $\overline{1,n}$.

\begin{enumerate}       
        \item Попередньо в КС формується та розміщується табличка \newline T = $\{D_{i}, H_{i}, i = \overline{1,n}\}$, де $H_{i} = h(p_{i},D_{i})$.
        \item (При контакті) $A_{i}: (p_{i},D_{i}) \rightarrow $ КС.
        \item КС перевіряє: ?$D_{i} \in T$? (ідентифікація),
        \item ?$h(p_{i}, D_{i}) = H_{i}? \longleftrightarrow ?H_{i} \in T$? (автентифікація).
        \item Якщо все виконується, то авторизація, інакше - відмова.  
\end{enumerate}
   
    \item \textit{Паралельна автентифікація за допомогою симетричного шифрування стійких до атак на основі відкритого тексту (ВТ).}
     \parКористувачі $A_{i}$ мають ідентифікатори $D_{i}$ та паролі $p_{i}$, $i = \overline{1,n}$.
     \parНехай $E_{k}$ - алгоритм симетричного шифрування, стійкий до атак на основі ВТ.
     
\begin{enumerate}
     \item Попередньо в КС формується та розміщується табличка 
     \item T = $\{D_{i}, C_{i}, i = \overline{1,n}\}$, де $C_{i} = E_{p_{i}}(D_{i})$.
     \item $A_{i}: (p_{i},D_{i}) \rightarrow $ КС.
     \item КС перевіряє: ?$D_{i} \in T$? та ?$E_{p_{i}}(D_{i}) = C_{i}$?
     \newline Якщо все виконується, то авторизація, інакше -- відмова.
\end{enumerate}

    \item \textit{Парольна автентифікація з захистом від коротких паролів.}
\begin{enumerate}
     \item модифікація в варіанті 1 з хеш-функцією h(x) - використання односторонніх хеш-функцій з секретним ключем (який відомий лише адмін-у КС).
     \item модифікація в варіанті 2: табличка T = $\{D_{i}, C_{i}, i = \overline{1,n}\}$ - використання секретного ключа k: $C_{i} = E_{p_{i}\oplus k}(D_{i})$.
     \newlineПеревірка аналогічна.
\end{enumerate}    
    
    \item \textit{Парольна автентифікація зі змінним паролем.}
    \par Користувачі $A_{i}$ мають ідентифікатори $D_{i}$ та паролі $p_{i}^{(0)}, i = \overline{1,n}$.
    \newline Наявна одностороння стійка до колізій хеш-функція h(x) (відкрита).
\begin{enumerate}
    \item Кожен $A_{i}$ обчислює ряд разових паролів: 
    \[
    p_{i}^{(0)}, p_{i}^{(1)} = h(p_{i}^{(0)}), p_{i}^{(2)} = h(p_{i}^{(1)}) = h^{2}(p_{i}^{(0)}) \vdots p_{i}^{(t)} = h^{t}(p_{i}^{(0)}) \: \text{та} \: p_{i}^{(t)} \rightarrow  \text{КС}, 
    i = \overline{1,n}.
    \]
    \item КС формує табличку T = $\{D_{i}, p_{i}^{(t)}, i = \overline{1,n}\}$.
    \item (При контакті) $A_{i}: (D_{i}, p_{i}^{(t-1)}) \rightarrow $ КС.
    \item КС перевіряє: ?$D_{i} \in T$? та ?$h(p_{i}^{(t-1)}) = p_{i}^{(t)}$?
    \newline Якщо все в виконується, то виконується авторизація і КС в табличці T змінює $(D_{i}, p_{i}^{(t)})$ на $(D_{i}, p_{i}^{(t-1)})$.
    \item (При другому контакті) $A_{i}: (D_{i}, p_{i}^{(t-2)}) \rightarrow $ КС.
    \item КС перевіряє: ?$D_{i} \in T$? та ?$h(p_{i}^{(t-2)}) = p_{i}^{(t-1)}$? 
\end{enumerate}   
   Якщо все виконується, то КС в табличці T змінює $(D_{i}, p_{i}^{(t-1)})$ на $(D_{i}, p_{i}^{(t-2)})$.
    \newline \textbf{\textit{Примітка:}} Якщо всі разові паролі будуть вичерпані, потрібно згенерувати нові їх послідовності.
    \newline Але що робити при технічному збої? Зрозуміло, що повторювати пароль не можна. Тому при, наприклад, однократному збої на кроці 6):
    $A_{i}: (D_{i}, p_{i}^{(t-3)}) \rightarrow $ КС. Після чого КС перевіряє: ?$D_{i} \in T$? та ?$h^{2}(p_{i}^{(t-3)}) = p_{i}^{(t-1)}$? після чого алгоритм продовжує свою роботу.
 \end{enumerate} 
 
 \item Автентифікація за допомогою симетричних криптосистем.
  \parКористувачі $A_{i}$, $i = \overline{1,n}$ мають ідентифікатори $D_{i}$ та $K_{ij}$ - секретні ключі $A_{i}$ та $A_{j}$ зі спільною системою симетричного шифрування $(E_{k}, D_{k})$.
  \begin{enumerate}
  
  \item Ініціатор аутентифікації $A_{i}: D_{i} \rightarrow A_{j}$.
  \item $A_{j}$ генерує випадкове $M \rightarrow A_{i}$.
  \item $A_{i}$ шифрує на спільному ключі повідомлення: $E_{K_{ij}}(M) = C_{ij} \rightarrow A_{j}$.
  \item $A_{j}$ перевіряє: $D_{K_{ij}}(C_{ij}) = M$.
  \end{enumerate}
  
 \item Автентифікація за допомогою асиметричних криптосистем (на прикладі RSA).
  \parКористувачі $A_{i}$, $i = \overline{1,n}$ мають відкриті $(n_{i}, e_{i})$ та закриті $d_{i}$ ключі RSA.
 \begin{enumerate}
   \item За допомогою цифрового підпису (ЦП).
  
   \begin{enumerate}
   \item $A_{i}$ генерує випадкове М і підписує його, тобто:
   \newline $S = M^{d_{i}}mod$ $n_{i}$, $\Rightarrow  (M, S) \rightarrow A_{j}$.
   \item $A_{j}$ перевіряє ЦП: ?$S^{e_{i}}mod \:n_i= M$?
   \item За допомогою шифрування.
   \end{enumerate}

 \end{enumerate}
 \item Автентифікація за допомогою протоколу доказу без розголошення.
\end{enumerate}