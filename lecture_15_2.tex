\section{Ядра важкооборотних функцій та генератори псевдовипадкових бітів}
\begin{flushright}
\emph{(Автор: Максим Хоменко. Трохи редагувалось.)}
\par \emph{(Версія від 29 січня 2017 р.)}
\end{flushright}

\subsection{Поняття ядра функції}\label{section:ker}

\begin{mydef}
Функція $f$ називається важкооборотною, якщо виконуються такі дві умови:

\begin{enumerate}
\item $y=f(x)$ ефективно обчислюється, тобто за поліноміальний час.
\item Жоден ефективний алгоритм для більшості аргументів $x \in \{0,1\}^n$ невзмозі за образом $y=f(x)$ знайти ніякого елемента $x'$ такого, що $f(x')=y.$
\end{enumerate}
\end{mydef}

Кандидатами у важкооборотні функції виступають \label{ex:func} 
\begin{enumerate}
\item[1)] Функція RSA: $f(x)=x^e \bmod m, \; \text{де} \; m=p \cdot q,\; p\neq q, \; (e,\varphi(m))=1$. 
\item[2)] Функція Ель-Гамаля: $f(x)=g^x \bmod p,\;\text{де}\; p-\text{просте},\; g \in \Zgroup{p} - \text{генератор}.$ 
\item[3)] Функція Рабіна: $f(x)=x^2 \bmod m ,\; \text{де}\; m=p \cdot q,\; p\neq q,\; p,q=4k+3.$ Числа виду $p$ та $q$ називають числами Блюма.
\end{enumerate}

Функції RSA і Рабіна є кандидатами у важкооборотні функції із секретом. \par
Секретом у RSA є $d=e^{-1} \bmod \varphi(m)$ , а у Рабіна -- це $p$ та $q$. \par
Функція Ель-Гамаля, як Ви вже зрозуміли, є кандидатом у односторонню функцію без секрету. 

\begin{remark}\
\begin{enumerate}
\item Функція RSA, при фіксованих $n,e$ є бієкцією. Тобто $f: \Zring{m} \rightarrow \Zring{m}$ -- бієктивне відображення.
\item Функція Рабіна є бієктивною, якщо розгяладати її при фіксованому $n$. \par
Тобто $f: \Qset{m} \rightarrow \Qset{m}$ -- бієктивне.
\item Функція Ель-Гамаля є бієктивною при фіксованих $p$,$q$. Тобто $f: \Zring{m} \rightarrow \Zring{m}$ -- бієктивне.
\end{enumerate}
\end{remark}

Узагальнемо усі випадки і розглянемо бієктивне відображення $f: D \rightarrow D$, де $D$ -- деяка множина. Водночас розгянемо ефективно обчислювальный предикат $B: D \rightarrow \{0,1\}$ -- для кожного $x \in D$ можна легко обчислити біт $B(x)$. \par 
Нехай $f(x)=y.$ Оскільки $f$ -- бієкція, то $B(x)$ одночасно визначається елементом $y$. Однак це не означає, що маючи $y$, отримати $B(x)$ легко. Якщо немає ефективного алгоритму, який би для більшості елементів $x\in D$ за заданим значення $f(x)$ давав біт $B(x)$, то предикат $B$ називається \textit{ядром} функції $f$. Ілюстрацію можно побачити на рисунку \eqref{label:illustration}.

\begin{figure}[h]
\begin{center}

\begin{tikzpicture}
\draw [thick, <->] (0,-2) node[below] {$B(x)$} -- (0,0) node [left] {$x$} -- (2,0) node [right] {$f(x)$};
\draw [thick, ->]  (2,-0.2) -- (0.2, -2);
\node at (1,0.25) {легко};
\node at (-0.85,-1) {легко};
\node at (2.15,-1) {важко};
\end{tikzpicture}

\end{center}
\caption{Важкооборотна функція $f$ та її ядро $B$.}
\end{figure}\label{label:illustration}

Нескладно збагнути, коли функція $f$ має ядро, то вона важкооборотна. Справді, якби був ефективний алгоритм для обчислення оберненої функції $f^{-1}$, то $B(x)$ було би легко отримати з $y$ композицією алгоритмів для обчислення $f^{-1}$ і $B$. \par 

Нехай $D_n$ -- послідовність множин, для якої $n = \lceil log||D_n||\rceil$, де $||D_n||$ -- довжина послідовності. 

\begin{mydef}
Поліноміально обчислювальний предикат $B: D_n \rightarrow \{0,1\}$ називається \textit{ядром} функції $f: D_n \rightarrow D_n$, якщо будь-який поліноміальний ймовірнісний алгоритм на вході $f(x)$, де  $x$ -- випадковий елемент із $D_n$, видає значення  $B(x)$ з імовірністю меншою, ніж $\frac{1}{2}+\frac{1}{n^c}$, для довільної константи  $c$ при досить великих  $n$. Ймовірність береться як за випадковим вибором  $x$, так і за випадковою послідовністю ймовірносного алгоритму.   \par 
\end{mydef}

Поліноміальна обчислювальність означає можливість обчислення за час, обмежений поліномом від $n$. \par 

По суті, означення говорить, що немає ніякого кращого способу, отримати $B(x) $ за $f(x)$, ніж просто взяти навмання випадковий біт. \par 

Для перелічених кандидатів у важкооборотні функції (\pageref{ex:func}) в якості ядра запропоновано наступні предикати.\par 

\label{eq:predicate}
Для функції RSA : $\Zring{m} \rightarrow \Zring{m}$ з фіксованими параметрами $e,m$, а також для функції Рабіна: $f : \Qset{m} \rightarrow \Qset{m}$ з фіксованим параметром $m$, $B$ є предикатом парності. Тобто 
\begin{equation}
B_1(x) = \left[x \divisible 2 \right] = \begin{cases}
        1, &\text{якщо $x$ парне} \\
        0, &\text{інакше}
        \end{cases}  \\
\end{equation}
де під $[.]$ розуміються \emph{дужки Айверсона}. \par 

Для функції Ель-Гамаля : $\Zgroup{p} \rightarrow \Zgroup{p}$ із фіксованими параметрами $p$ і $q$, предикат $B$ задається співвідношенням:
\begin{equation}
B_2(x) = \left[x \leq \tfrac{p-1}{2} \right] = \begin{cases}
        1, &\text{якщо }x \leq \frac{p-1}{2}  \\
        0, &\text{інакше}
        \end{cases}  \\
\end{equation}
Цей предикат також називають -- предикат половинчастості. \par 
Застосування важкооборотних функцій та їх ядер для генерування послідовностей псевдовипадкових бітів будет описано у розділі \eqref{section:psevdo_gener}.
 
\subsection{Псевдовипадкові генератори із важкооборотних функцій}\label{section:psevdo_gener}

Нехай $f$ -- важкооборотна функція, яка після фіксації відповідних параметрів є бієкцією на множині $D_n$, причому $\lceil log||D_n|| \rceil = n$. Приклади дивіться у розділі \eqref{section:ker}.  \par
Нехай $B$ -- ядро функції $f$(відповідні предикати дивіться у розділі \eqref{section:ker}).  Використовуючи $f$ і $B$, сконструюємо генератор $G$ з розширенням $l(n)$, де $l(n)$ обмежена деяким поліномом від $n$. Вважається, що $D_n \subset \{0,1\}^n$, і що належність до цієї множини можна ефективно розпізнавати.  

\begin{algorithm}
Вхід: паросток $x_0 \in \{0,1\}^n$.
\begin{itemize}
\item Якщо $x_0 \not\in D_n$, то зупинитись, інакше продовжувати. 
\item Для $i = \overline{1, l(n)}$ обчислити $\sigma_i = B(x_{i-1})$ і $x_i=f(x_{i-1})$.
\item Подати на вихід последовність $\sigma_1\sigma_2\cdots\sigma_{l(n)}$.
\end{itemize}
\end{algorithm}


\begin{theorem}\label{theorem:1}
Для кожної важкооборотної функції $f$ і її ядра $B$, описаний генератор $G$ є псевдовипадковим. 
\end{theorem}

\begin{remark}
Теорема говорить, що якщо паросток $x_0$  вибирається рівноймовірно з $\{0,1\}^n$, то ніякий поліноміальний ймовірнісний алгоритм неспроможний відрізнити послідовність $\sigma_1\sigma_2\cdots\sigma_l $ від справжньої випадкової послідовності такої ж довжини. Насправді це твердження залишається в силі, навіть коли оприлюднюється останній елемент  $x_l$, тобто, послідовність $\sigma_1\sigma_2\cdots\sigma_l x_l$ не можна відрізнити за поліноміальний час від послідовності $\rho_1\rho_2\rho_l x_l$, де $\rho_1,\cdots,\rho_l$ -- випадкові і незалежні між собою біти. Зауважимо також, що $x_l$ є випадковим елементом множини $D_n$.
\end{remark}

\subsection{BBS генератор}

Генератор на основі функції Рабіна називають генератором BBS, де абреавіатура утворена від імен його винахідників -- Ленори Блюм, Мануїла Блюма та Майкла Шуба. \par
Нехай параметр $l=l(n)$ обмежений поліном від $n$. Щоб отримати із $n$ випадкових бітів $l$ псевдовипадкових, слід:
\begin{itemize}
\item вибрати випадково просте $p$ та $q$ такі, що $2^{n-1} < m < 2^n$ для $m=pq$, і $p\equiv q \equiv 3(\bmod 4)$;
\item вибрати випадковий елемент $r \in \Zring{m}$ і обчислити $x_0=r^2 \bmod m$ ;
\item для $i$ від 1 до $l$ обчислити: 
\[
\sigma_i=x_{i-1} \bmod 2, x_i=x^2_{i-1} \bmod m;
\]
\item подати на вихід послідовність $\sigma_1\sigma_2\cdots\sigma_l $.
\end{itemize}

Нескладно помітити, що ми притримувались загальної конструкції попереднього пункту із функцією $f(x) = x^2 \bmod m$ і її ядром, предикатом парності $B(x)=x \bmod 2$. Нагадаємо, що функція $f$ не є бієктивною на $\Zring{m}$, але є такоє на $\Qset{m}$. Саме тому в якості паростка ми вибирали $x_0=r^2 \bmod m$ -- випадковий елемент на множини $\Qset{m}$. За теоремою \eqref{theorem:1} генератор BBS є псевдовипадковим, а отже, криптографічно надійним (за умови, що функція Рабіна, справді є важкооборотною).
 
\begin{claim}
За $x_l$ і простими $p$ та $q$ можна ефективно визначити всю послідовність $x_0, \cdots , x_{l-1}$, а отже і послідовність $\sigma_1 \cdots \sigma_l$.
\end{claim} 

Зауважимо, що твердження не суперечить криптографічній надійноссті генератора BBS, оскільки в застосуваннях прості $p$ і $q$ тримаються у таємниці.

\begin{proof}
Припустимо, що відомо елемент $x_i$, де $1\leq i \leq l$, і покажемо, як отримати $x_{i-1}$. За бієктивністю функції $f(x)=x^2 \bmod m$ на $\Qset{m}$, елемент $x_{i-1}$  однозначно визначається такими умовами:
\begin{enumerate}
\item $x^2_{i-1} \equiv x_i (\bmod\: m)$,
\item $x_{i-1} \in \Qset{m}$.
\end{enumerate}
Введемо позначення $y=x_{i-1} \bmod p$ і $z=x_{i-1} \bmod q$. За наслідком із Китайської теореми про остачі, лишки $y$ і $z$ задовольняють умови:
\begin{align}\label{eq:1}
y^2 &\equiv x_i (\bmod\: p) & z^2 &\equiv x_i (\bmod\: q) 
\end{align}
\begin{align}\label{eq:2}
y & \in \Qset{p} & z & \in \Qset{q} 
\end{align}
і, більше того, така пара $y$ та $z $ однозначно визначає $x_{i-1}$, причому ефективним чином. \par
Отже, досить знайти $y $ та $z$. Для цього виктористаємо співвідношення:

\begin{equation}\label{eq:3}
y = x^{\frac{p+1}{4}}_i \bmod \: p \; \text{та} \; z=x^{\frac{p+1}{4}} \bmod \: q .
\end{equation}
З них беспосередньо видно, що $y$ і $z$ ефективно обчислюються із застосуванням бінарного методу піднесення до степеня. \par 
Нам залишилось обгрунтувати рівність \eqref{eq:3}. Доведемо першу з них (для другої аргументи ідентичні). Досить перевірити умови \eqref{eq:1} та \eqref{eq:2} для $y$, заданого співвідношенням \eqref{eq:3}. Умова \eqref{eq:2} негайно випливає із замкнутості $\Qset{p}$ відносно множення. Умова \eqref{eq:1} випливає з того, що добування кореня за модулем $p$ таким, що $p \equiv 3(\bmod \: 4)$, еквівалентне піднесенню до степеня $\frac{p+1}{4}$. 

\end{proof}
