\section{Криптосистемы на эллиптических кривых}
\begin{flushright}
\emph{(Автор: Семен Лигін. Російською мовою. Не редагувалось.)}
\par \emph{(Версія від 18 січня 2017 р.)}
\end{flushright}


На эллиптические системы переносятся только задачи, основанные на задаче дискретного логарифма

\begin{equation}
    y= \alpha^x mod p
\end{equation}
где $\alpha$ - примитивный элемент над $F_p$

Недостатки: зыбкость теоретического обоснования, медленная работа. 

Первое понятно, по поводу второго: вот временн\'{а}я сложность. 
    $$T(n) = e^{cn^\nu(logn)^{1-\nu}}$$
    $$0<=\nu<=1$$
    $$\nu=0:$$
    $$T(n) = e^{clogn}=n^c$$
    $$\nu = 1:$$
    $$T(n) = e^{cn}$$
Если строго от 0 до 1 - то такие алгоритмы субэкспоненциальные
Вообще ню примерно равно 1/3
В связи с этим создатели криптосистемы вынуждены увеличивать длину ключа, что не ускоряет работу системы. 
Сейчас длина ключи порядка 2048 бит. 

В 1985 году Миллер и Коблитц предложили строить криптосистемы на эллиптических кривых. А именно: было предложено задавать на эллиптических кривых аддитивную абелеву группу. Собственно:

E - эллиптическая кривая,
P - точка кривой большого порядка, под порядком имеем в виду порядок в группе.

$Q=xP=\underbrace {P+...+P}_{\text x}$

И зная P, Q и E, затруднительно найти Х.

Мультипликативная группа всегда циклична. А группа точек в Е - ваще хз какая, любая может быть. Но это нормально. 

Кстати, $\alpha$ - не обязательно должно быть примитивным элементом. Альфа должно иметь такой порядок, что невозможно создать таблицу для всех степеней $\alpha$, примитивно говоря. В конечном поле легко найти нужные элементы, вот их и берём. 
На практике берём кривые, порядки которых делятся на большое, просто n.

То есть, точки порядка эн. 

и: 
$P, 2P, 3P, ... , nP = O_e$ 

- получается циклическая группа. 

А это - задача дискретного логарифма (найти x по P,Q,E)

Задача дискретного логарифма на эллиптических кривых труднее, чем в конечном поле. Там мы работаем с простыми числами и неприводимыми полиномами, а на эллиптических кривых такого нет, следовательно, криптосистемы на эллиптических кривых при той же длинне ключа l более стойкие, чем в конечном поле. 

В конечном поле ключи длины около 1024B, на ЭК - порядка 160B.

Но не всё так просто. В частности, схема Диффи-Хеллмана не переносится на ЭК тривиально, есть проблемы. 

Вот есть алгоритм Диффи-Хеллмана:

A и B выбрали себе $p$ и $\alpha$ - примитивный элемент $F_p$

$1 < K_A < p-1$

$1 < K_B < p-1$

$y_A = \alpha^{K_A} mod p   -> y_A^{K_B}mod p = \alpha^{K_A * K_B} = K$

И в то же время B:

$y_B = \alpha^{K_B} mod p   -> y_B^{K_A}mod p = \alpha^{K_A * K_B} = K$

И получают одинаковое K.

На эллиптических кривых он же:

$F_q, E, P: ord P = n$

$1 < K_A < n$

$1 < K_B < n$

$Q_A = \alpha^{K_A}*P   -> y_{K_B}*Q_A = K_A * K_B * P = K$

$Q_B = \alpha^{K_B}*P   -> y_{K_A}*Q_B = K_A * K_B * P = K$

Кажется, что они тривиально переносятся. Но есть проблемы. 

КС на ЭК были предложены в 1985 году, но стандарты на них появились около 1999 года. 
Вот в схеме Диффи-Хеллмана: выбираем E, в которой есть точка большого порядка. А как выбрать-то?Нельзя просто посчитать руками. 

Есть алгоритм Скуффа(Schoof)

$P \in Poly$, но если кривая над $F_q$, то $O((log9)^9)$

Многовато.

Также известно неравенство Хассе для определения порядка эллиптической кривой. 
\begin{theorem}[Неравенство Хассе]

Пусть N - порядок эллиптической кривой. 

E - кривая над $F_q$

Тогда
$$ q+1-2*(q)^{-1/2}<=N_E<=q+1+2*(q)^{-1/2} $$
\end{theorem}

\begin{proof}
Поле берём характеристики не два и не три. Подставляем Х и смотрим на квадратичный вычет. 

В среднем половина вычетов и половина невычетов. 

Это $q/2$, в каждом есть по $2y$ - это единица, и $2q^{1/2}$ - это дисперсия.

\end{proof}

Есть масса способов получить одну любую ЭК, например:

$y^2 = x^3 + ax + b$ над $F_q, a,b \in F_q$

Кривую рассмотрим над ${F_q}^n$
Правда, и кривая тогда имеет больше точек. 

Рассмотрим такую кривую. В неравенстве Хассе числа в промежутке лежат очень узко (в каком-то смысле, мне непонятном)

Пусть N - число из интервала, заданного неравенством Хассе. 
Пусть $N_1$ - порядок кривой с A и B.

\begin{table}[]
    \centering
    \begin{tabular}{c|c}
        $W F_Q {F_Q}^2 ... {F_Q}^r$ \\
        $   N_1 N_2 .....N_r$
    \end{tabular}
    \caption{Такую табличку строим}
    \label{tab:my_label}
\end{table}

И есть итеративные формулы, чтобы считать $N_i$ из $N_j$.

Нашли $N_1$, по нему считаем $N_2$ и проверяем, что $N_2$ делит $P$, где P - большое простое число.

Чуть-чуть подробнее об этой проверке. "При помощи кофактора, С - маленькое $z^*$"

Число делим на маленькие числа и затем проверяем на простоту. 
Да - ок. 
Нет - берём следующее число. 

Перебирая находим кривую из $F_q$, и на ${F_q}^n$ будет тот же порядок. 

Теперь ещё раз. Есть уравнение кривой. Рассмотрим её над $F_q$. А надо построить над ${F_q}^n$. 

$$y^2 = x^3 + ax + b$$
$$F_q \to {F_q}^r$$

Знаем, что 
$$ q+1-2*(q)^{-1/2}<=N_E<=q+1+2*(q)^{-1/2} $$

И что:
$$N_1 \to N_2 \to N_3 \to ... \to N_r$$

W - первое число из ***. Оно соответствует какой-то кривой. 

Считаем $N_r$

Проверяем, делится ли $N_r$ на большое простое число. 

Также следует понимать, что в маленьком поле мы можем просто непосредственно посчитать порядок кривой. 

Схема шифрования Эль-Гамаля

Обычно:

$A: p, \alpha$

$1<k<p-1$ - секретный ключ

$y = \alpha^k mod p$ - открытый ключ

$B: m<p$

$1 < X_m < p-1$ - генерится каждый раз заново

$С_1 = \alpha^{X_m} mod p$

$С_2 = y^{X_m}*m mod p$

$B: (C_1, C_2) -> A $

$A: (C_1 ^ {-k} * C_2 = \alpha^{-k*X_M} * \alpha^{k*X_m}*m mod p = m$

На эллиптических кривых есть проблема: мы шифруем точку эллиптической кривой, а надо шифровать число. И нет такого детерменированного алгоритма, который сможет их сопоставить. 

$$ 0 < m < M $$

Пусть - над полем характеристики q, не 2 и не 3. 

M-1 - значение наибольшего сообщения. 

$ M - 1 < q$

$k : 1/{2^k}$ - вероятность не сопоставить точку. 

Каждому M можем сопоставить отрезок. Берём точку и подставляем правую часть ($x^3+ax+b$)

Это и будет точкой эллиптической кривой $m -> P_m$

Если квадратичный невычет - то берём следующее число в этом отрезке. 

И тогда: пусть $1/2$, что квадратичный вычет.

Тогда $1/2^k$ - вероятность не найти. Это ок. 

$A: F_q, E, P$(большого порядка),$ord P = N$
$0<k<n$ - k - секретный ключ. Q - открытый ключ, $Q = k*P$

$B: 0 < X_m < n, m -> P_m$,
последнее - сопоставляет точку. 

$C_1 = x*p $

$C_2 = X_m*Q + p_m$

$A: (-k)*C_1 + C_2$

$(-k)*p = k*(-p)$

- взаимная однозначность
Например, 100p надо считать:

$100 = 2^6 + 2^5 + 2^2$

$p,2p,2^2p,2^3p,2^4p,2^5p,2^6p$

И складываем:

$2^2p + 2^5p + 2^6p = 100p$ - схема Горнера

Важно. Вот допустим есть у нас формулы сложения эллиптической кривой. Но ведь там же есть деление. Соответственно, кривую рассматривают в других координатах, например, в проективных. 

А именно:

$F_q$            $(\lambda*x,\lambda*y,\lambda*z)$

x, y, z и $\lambda$ - элементы поля $F_q$

Между координатами в этих системах есть соотношение:

$(\lambda*x,\lambda*y,\lambda*z)->(x/z, y/z, 1)$

И, если есть уравнение эллиптической кривой $y^2 = x^3 + ax + b $

То в проективных координатах:

$${y/z}^2={x/z}^3+ax/z+b$$

$$y^2*z=x^3+ax*z^2+b*z^3$$

А это уравнение однородное. 

Это позволяет выписать формулы для точек без деления. 

А вообще есть и ещё другие координаты: смешанные - удвоение в одних, а сложение в других. 

А ещё есть алгоритм Ито-Цудзии. 

А ещё удобно делать некоторые операции в ОНБ.
