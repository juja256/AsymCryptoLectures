\section{Алгоритми ймовірнісного шифрування}
\begin{flushright}
\emph{(Автор: Андрій Турчин. (Трохи редагувалось))}
\par \emph{(Версія від 22 січня 2017 р.)}
\end{flushright}

\subsection{Побудова системи ймовірнісного шифруванняна псевдоквадратах користувача А}

\begin{mydef}
\emph{Псевдоквадратом за модулем $n$}, де $n$ - складене число, називається таке число
$a \in \mathbb{Z}$, що $\gcd(a, n) = 1$ та символ Якобі $\Legendre{a}{n} = 1$, але при 
цьому $a$ не є квадратним лишком.
\end{mydef}

Нехай $n = p \cdot q$, $p \neq q$ -- прості. Позначимо через $\Jset{n}$ множину 
$$\Jset{n} = \left\{a \in \Zgroup{n} \colon \Legendre{a}{n} = 1 \right\}.$$
Тоді маємо $\Jset{n} = \Qset{n} \cup \PQset{n}$,  $\Qset{n} \cap 
\PQset{n} = \emptyset$, де $\Qset{n}$\ - множина кв. лишків, 
$\PQset{n}$ - множина псевдоквадратів. При цьому 
$|\Qset{n}| = |\PQset{n}|$.

Для побудови системи імовірнісного шифрування виконаємо наступні кроки.
\begin{enumerate}
\item Вибираємо великі прості $p \neq q$. Відповідно обчислюємо $n = p 
\cdot q$.

\item Знаходимо псевдоквадрат $a$ наступним чином:

\begin{enumerate}
\item обираємо квадратичний нелишок $a_{1}$ за модулем $p$ (шляхом перевірки співвідношень $\Legendre{a_{1}}{p} = -1$ 
або $a_{1}^{\frac{p-1}{2}} = -1$;
\item аналогічно обираємо квадратичний нелишок $a_{2}$ за модулем $q$;

\item знаходимо розвязок системи:
\begin{equation*} 
\begin{cases}
x =\ a_{1} \bmod p \\
x =\ a_{2} \bmod p
\end{cases}
\end{equation*}
При цьому 
$$\Legendre{a}{n} = \Legendre{a}{p} \Legendre{a}{q} = \Legendre{a_{1}}{p} \Legendre{a_{2}}{q} = (-1)(-1) = 1,$$
і a не є квадратичним лишком за модулем $n$.
\end{enumerate}

\item  Оголошуємо відкритий ключ $(n,a)$ для зашифрування повідомлень абоненту А; при цьому пара $(p,q)$ -- це секретний ключ А для розшифрування.
\end{enumerate}

\textbf{Зашифрування:}

Відкритий текст довжиною $M < n$, \todo{$M$ - містить користувачів B}
\begin{enumerate}
\item Обираємо послідовність випадкових чисел $r_{i}\colon 1<r_{i}<n, i = \overline{1,t}$ 

\item Для всіх значень $i = \overline{1,t}$ обчислюємо
\begin{equation*}
C_{i}= 
\begin{cases}
r_{i}^{2} \bmod n, &\text{якщо $m_{i}$ = 0}\\
a r_{i}^{2} \bmod n, &\text{якщо $m_{i}$ = 1}
\end{cases}
\end{equation*}

\item Формуємо шифртекст: $C = C_{1}C_{2}...C_{t}$, де $1<C_{i}<n$.
\end{enumerate}

\textbf{Розшифрування:}

Для всіх значень $i = \overline{1,t}$ обчислюємо:
\begin{equation*}
m_{i}= \left[ C_i \in \Qset{n}\right] =
\begin{cases}
        0, &\text{якщо $C_{i}$ - кв. лишок}\\
        1, &\text{якщо $C_{i}$ - кв. нелишок}
\end{cases}
\end{equation*}
Відповідно, $M = m_{1}m_{2}...m_{t}$.

\subsection {Алгоритм ймовірнісного шифрування на основі RSA}

Нехай користувач А має RSA із відкритим ключем $(n,e)$ і секретним ключем 
$d$.

\textbf{Зашифрування ВТ користувачем користувачем B для користувача А:} 

\begin{enumerate}
        \item Представляємо $M = m_{1}m_{2}...m_{t}$, $m_{i} \in {0,1}$, $M<n$ ;
        \item Генеруємо $t$ незалежних випадкових чисел $r_{i} \in_R 
        \Zgroup{n}$, які задовольняють умові:
\begin{equation*}
        r_{i}= 
        \begin{cases}
        &\text{парне, якщо }m_{i} = 0\\
        &\text{непарне, якщо }m_{i} = 1
        \end{cases}
\end{equation*}

        \item Обчислюємо для всіх $i = \overline{1,t}$: $C_{i} = r_{i}^{e} \bmod n$;
        \item Складаємо шифротекст $С = C_{1}C_{2}...C_{t}$, який 
        надсилається абоненту А.
\end{enumerate} 

\textbf{Розшифрування ШТ користувачем А:}
\begin{enumerate} 
\item Для всіх $i = \overline{1,t}$ обчислюємо біт $m_{i} = 
\left(c_{i}^{d}\bmod n \right) \bmod 2$.

\item Складаємо відкритий текст $M = m_{1}m_{2}...m_{t}$.
\end{enumerate}
Cтійкість даної схеми імовірнісного шифрування грунтується на стійкості алгоритму RSA.

