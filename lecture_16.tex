

\section{Алгоритм ймовірнісного шифрування RSA}
\begin{flushright}
\emph{(Автор: Андрій Турчин. (Трохи редагувалось))}
\par \emph{(Версія від 22 січня 2017 р.)}
\end{flushright}

\mydef Псевдоквадратом a по mod n, де n - складне число, називається таке a $\in$\ $\mathbb{Z}$\ , що (a,n) = 1, а символ Якобі ($\frac{a}{n}$)\ = 1, при цьому a не являється квадратним лишком   

Нехай n = p$\cdot$q, p$\neq$q - прості. Позначимо $\mathbb{G}_{n}$\ - множину тих a:\{a $\in$\ $\mathbb{G}^{*}_{n}$ : ($\frac{a}{n}$)\ = 1\}

${\mathbb{G}_{n}}$\ = ${\mathbb{Q}_{n}}$\ $\cup$ $\tilde{\mathbb{Q}_{n}}$,  ${\mathbb{Q}_{n}}$\ $\cap$ $\tilde{\mathbb{Q}_{n}}$ = 0, ${\mathbb{Q}_{n}}$\ - множина кв. лишків, $\tilde{\mathbb{Q}_{n}}$ - множина псевдоквадратів. При цьому |${\mathbb{Q}_{n}}$| = |$\tilde{\mathbb{Q}_{n}}$| 

\subsection{Побудова системи ймовірнісного шифруванняна псевдоквадратах користувача А}

\begin{enumerate}
\item Вибираємо великі прості p $\neq$ q. Відповідно обчислюємо n = p$\cdot$q;

\item Знаходимо псевдоквадрат a наступним чином:

a) \ вибираємо квадратний нелишок $a_{1}$ по mod(p) або ж обчислюють ($\frac{a_{1}}{p}$)\ = -1, 

або $a_{1}^{\frac{p-1}{2}}$ = -1

b) \ вибираємо $a_{2}$ нелишок по mod(q) ($\frac{a_{2}}{p}$)\ = -1

c) \ знаходимо розвязок системи:
\begin{equation*} 
\begin{cases}
x =\ a_{1} \bmod p \\
x =\ a_{2} \bmod p
\end{cases}
\end{equation*}
При цьому ($\frac{a}{n}$)\ = ($\frac{a}{p}$)($\frac{a}{q}$)\ = ($\frac{a_{1}}{p}$)($\frac{a_{2}}{q}$)\ = (-1)(-1) = 1 і a не являється квадратичним лишком

\item  Оголошуємо відкритий ключ (n,a) для зашифрування повыдомлення для А, при цьому (p,q) - це секретний ключ А для розшифрування.
\end{enumerate}

\textbf{Зашифрування:}

ВТ довжиною M < n, M - містить користувачів B
\begin{enumerate}
\item Вибираємо послідовність випадкових чисел $r_{i}$: 1<$r_{i}$<n i = $\overline{\rm 1,t}	$ 

\item $\forall i = \overline{\rm 1,t}$ обчислюють 
\begin{equation*}
С_{i}= 
\begin{cases}
r_{i}^{2} \bmod n, &\text{якщо $m_{i}$ = 0}\\
a r_{i}^{2} \bmod n, &\text{якщо $m_{i}$ = 1}
\end{cases}
\end{equation*}

\item Формуємо ШТ: C = $C_{1}C_{2}...C_{t}$, де 1<$C_{i}$<n 
\end{enumerate}

\textbf{Розшифрування}

$\forall i = \overline{\rm 1,t}$ провіряє:
\begin{equation*}
m_{i}= 
\begin{cases}
	0, &\text{якщо $C_{i}$ - кв. лишок}\\
	1, &\text{якщо $C_{i}$ - кв. нелишок}
\end{cases}
\end{equation*}
Звідси, M = $m_{1}m_{2}...m_{t}$

\subsection {Алгоритм ймовірнісного шифрування за допомогою RSA}

Нехай користувач А має RSA p відкритим ключем  (n,e) і секретним ключем d

\textbf{Зашифрування ВТ користувачем користувачем B для користувача А:} 

\begin{enumerate}
	\item Уявляємо M = $m_{1}m_{2}...m_{t}$, $m_{i} \in {0,1}$, M<n ;
	\item Генеруэмо  t незалежних випадкових чисел $r_{i}$ $r_{i} \in {1,2,...,n-1}$
	При цьому 
\begin{equation*}
	r_{i}= 
	\begin{cases}
	&\text{парне, якщо $m_{i}$ = 0}\\
	&\text{непарне, якщо $m_{i}$ = 1}
	\end{cases}
\end{equation*}

	\item Обчислюємо для i = $\overline{\rm 1,t}$: $C_{i} = r_{i}^{e}\bmod n$;
	\item Складаємо С = $C_{1}C_{2}...C_{t}$ $\rightarrow A$;
\end{enumerate} 

\textbf{Розшифрування ШТ користувачем А:}
\begin{enumerate} 
\item Обчислити для i = $\overline{\rm 1,t}\ m_{i} = (c_{i}^{d}\bmod n)\bmod 2$

\item Складаємо ВТ M = $m_{1}m_{2}...m_{t}$
\end{enumerate}
Cтійкість схеми основана на стійкості RSA.

