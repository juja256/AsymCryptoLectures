\section{Системи шифрування та цифровий підпис Рабіна}
\begin{flushright}
\emph{(Автор: Маша Коляда. Редагувалось)}
\par \emph{(Версія від 22 січня 2017 р.)}
\end{flushright}

Дана система шифрування та цифрового підпису використовує односторонню функцію Рабіна : $y=x^2 \bmod n$, де $n=p \cdot q$, $p\ne q$ -- великі прості числа.

\textit{Побудова системи шифрування Рабіна користувачем А}:
\subsection{Схема шифрування Рабіна 1}

\begin{center}
 \textbf{Зашифрування відкритого тексту користувачем \textsl{B}}
\end{center}

\begin{enumerate}
        \item Користувач \textsl{А} обирає прості числа $p,q : p\ne q$. Зазвичай $p=2 \cdot p’+1$,  $q=2 \cdot q’+1$, де $p’, q’$ -- великі прості числа.
        \item Користувач \textsl{А} обчислює $n=p \cdot q$, де $p, q$ -- його секретний ключ.
        \item Користувач \textsl{A} \textit{«оголошує»} відкритий ключ $n$ для зашифрування повідомлень до нього.
        \item Зашифрування повідомлення \textsl{M} користувачем \textsl{B}: $ C=M^2 \bmod n$ – шифртекст, \par
         де $\sqrt{n}<M<n$.
        \item Користувач \textsl{B} відправляє повідомлення користувачу \textsl{А}.    
\end{enumerate}
\textbf{Зауваження}:  При $M^2<n$, розрахунок квадратного кореня аналогічний до розрахунку квадратного кореня на множині дійсних чисел, що легко виконується.\\

\begin{center}
\textbf{Розшифрування шифртексту користувачем  \textsl{А}}:
\end{center}

\begin{enumerate}
        \item   Використовуючи $p$, обчислює: $\pm \beta_p=C^{\frac{1}{2}}\bmod p$.
        \item Використовуючи $q$, обчисює: $\pm \beta_q=C^{\frac{1}{2}}\bmod q$.
        \item За китайською теоремою про лишки знаходить 4 корені:
        \[
         C^{\frac{1}{2}}\bmod n={(\pm \beta_p)\cdot q \cdot q^{-1}\bmod p 
         \pm \beta_q \cdot p \cdot p^{-1}\bmod q}.
        \]
        Отримує  ${M_1,M_2,M_3,M_4}$, після чого обирає один з цих коренів за змістом.   
\end{enumerate}


\subsection{Схема шифрування Рабіна 2(коли n – число Блюма)}

\begin{center}
 \textbf{Зашифрування  відкритого тексту $M$ : $\sqrt{n}<M<n$ користувачем \textsl{В}}
\end{center}
\begin{enumerate}
        \item Обчислює $M^2\bmod n=C$.
        \item Обчислює 2 біти: 
\begin{align*}
      b_1 &= \begin{cases}
        0, &\text{якщо \textsl{M} парне} \\
        1, &\text{інакше}
        \end{cases}  \\
      b_2 &=  \begin{cases}
         0, &\text{якщо} \left(\frac{M}{n}\right)=-1\\
         1, &\text{якщо} \left(\frac{M}{n}\right)=1
        \end{cases}
\end{align*}
        %\todo{(ПОМИЛКА. Оточення cases не працює в цьому фрагменті -- розібратись. -- С.Я.)}
        \item Формує зашифроване повідомлення $(C,b_1,b_2)$ та відправляє його користувачу \textsl{А}.   
\end{enumerate}

\begin{center}
 \textbf{Розшифрування  $(C,b_1,b_2)$ користувачем  \textsl{А}}
\end{center}

\begin{enumerate}
        \item  Аналогічно схемі шифрування Рабіна 1 знаходить ${M_1,M_2,M_3,M_4}$.
        \item Використовуючи $(b_1,b_2)$,  обчислює $ (b^i_1, b^i_2) , i=\overline{1,4}$ та обирає $M_i$ у якого $(b^i_1, b^i_2)= (b_1,b_2)$.    
\end{enumerate}
\ Стійкість до атаки на основі шифртексту повністю заснована на(еквівалентна) складності задачі факторизації. Схеми шифрування Рабіна 1 і 2 не стійкі до атаки на основі вибраного шифртексту.

\subsection{Атаки на схеми шифрування}
\begin{center}
\textbf{Атака на схему шифрування Рабіна 1}
\end{center}

\begin{enumerate}
        \item Криптоаналітик \textsl{Е} підбирає послідовність
        відкритого тексту $M_1$….$M_k$, обчислює відповідні їм  $C_1$…$C_k$.
        \item Відправляє отриманий шифртекст \textsl{A}.
        \item  $A$ : $M'_1$….$M'_k$. Перевіряє $M_i=\pm M'_i$.
        \item Якщо така пара знайшлась, то обчислює $\left(M_i\pm M'_i,n\right)=p \: \text{або} \:q$.     
\end{enumerate}

\begin{center}
\textbf{Атака на схему шифрування Рабіна 2}
\end{center}

\begin{enumerate}
        \item  Криптоаналітик \textsl{E} обирає відкритий текст \textsl{M}. Обчислює ($b_1,b_2$).
        \item Обчислює $M^2\bmod n=C$.
        \item Відправляє шифртекст ($C, b_1,b^{'}_2$), де $b^{'}_2=b_2 \oplus 1$
        \item Отримавши відкритий текст $M^{'}$, знає, що $M\ne M^{'}$, обчислює $\left(M_i\pm M^{'}_i,n\right)=p \: \text{або} \: q$.
\end{enumerate}


\subsection{Схема цифрового підпису Рабіна}

\begin{center}
\textbf{Побудова схеми цифрового підпису Рабіна користувачем А}
\end{center}

\begin{enumerate}
        \item Обирає великі прості числа $p,q$ : $p\ne q$.
        \item Обчислює геш-функцію $h(x)$.
        \item Оголошує $n=p \cdot q$.
        \item Оголошує відкритий ключ ($n, h(x)$) для перевірки цифрового підпису \textsl{А}.     
\end{enumerate}
\ При цьому $p,q$ -- секретний ключ користувача \textsl{А} для формування цифрового підпису.

\begin{center}
\textbf{Формування цифрового підпису А}
\end{center}

\begin{enumerate}
        \item   $M =m_1m_2…m_k$. Генерує випадкову послідовність $R=r_1…r_s$.
        \item   Обчислює $h(M||R)=h(m_1…m_k+r_1…r_s)=H$
        \item Перевіряє, чи є $H$ квадратичним лишком за модулем $n$.
        \begin{enumerate}
                \item Якщо виконується $H^{\frac{p-1}{2}}\bmod p=1$ і $H^{\frac{q-1}{2}}\bmod q=1$, переходимо до кроку 4.
                \item інакше переходимо до кроку 1.
        \end{enumerate}
\ Якщо n--число Блюма, то в средньому, цей пункт виконується на 4-ій спробі.
    \item       Обчислює $H^{\frac{1}{2}}\bmod n$:
    \begin{enumerate}
    \item       Використовуючи $p$ обчислює: $\pm \beta_p=H^{\frac{1}{2}}\bmod p$
    \item Використовуючи $q$ обчислює: $\pm \beta_q=H^{\frac{1}{2}}\bmod q$
    \item   За китайською теоремою про лишки знаходить 4 корені: 
\[    
    C^{\frac{1}{2}}\bmod n={\pm \beta_p \cdot q \cdot q^{-1}\bmod p \pm ( \beta_q) \cdot p \cdot p^{-1}\bmod q} \rightarrow M_1,M_2,M_3,M_4
\]
\end{enumerate}
   \item Обирає один з коренів $\beta$, та формує підпис ($M,R, \beta$)
\end{enumerate}

\begin{center}
\textbf{Перевіка цифрового підпису користувачем В}
\end{center}

\begin{enumerate}
        \item   $(M,R,\beta) \rightarrow M,R,\beta$
        \item Обчислює  $h(M||R)=h(m_1…m_k+r_1…r_s)=H$
        \item Перевіряє рівність : $\beta^2 = H \bmod n$.
        \begin{enumerate}
                \item Якщо виконується, то підпис правильний
                \item Інакше, не правильний.
        \end{enumerate} 
\end{enumerate}
\ Складність взлому цифрового підпису Рабіна поліноміально еквівалентна складності задачі факторизації.