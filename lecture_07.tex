\section{Одностороння функція Рабіна з секретом}
\begin{flushright}
\emph{(Автор: Антон Карпець.  Редагувалось.)}
\par \emph{(Версія від 22 січня 2017 р.)}
\end{flushright}

\subsection{Визначення}

\begin{mydef}
Односторонньою функцією Рабіна називається функція: 
\begin{equation} \label{eq:7.1}
y = x ^{2} \bmod n,
\end{equation}
де $n = p \cdot q,\; p \neq q$ ~--- великі прості числа. Секретом функції Рабіна є числа $p$ та $q$.
\end{mydef}

\begin{remark}
Одностороння функція Рабіна не є частковим випадком односторонньої функції RSA ($y = x ^{e} \bmod n$,
$n = p \cdot q,\; p \neq q$ ~--- великі прості числа), оскільки в схемі RSA $(e, \varphi(n)) = 1$, тобто, $(e, (p-1)(q-1)) = 1$. Це означає, що $e$ не ділиться націло на 2.
\end{remark}

\begin{remark}
Функція Рабіна не є бієкцією на множині $X = \{ 0, 1, ..., n-1 \}$ на відміну від RSA, яка є бієктивним відображенням на множині $X$. Окрім того, значення даної функції є квадратичними лишками, якщо $(x, n) = 1$.
\end{remark}

\begin{remark}
Функція, обернена до функції \eqref{eq:7.1} - $y ^{\frac{1}{2}} \bmod n$, яка називається дискретним добуванням квадратного кореня, обчислюється лише для тих $y$, які є квадратичними лишками і не є однозначною.
\end{remark}

\subsection{Складність обчислень}

\begin{enumerate}

\item Для $\forall x \in X = \{ 0, 1, ..., n-1 \}$ обчислення $y = x ^{2} \bmod n$ потребує однієї операції піднесення до квадрату за модулем $n$.
\item Секретом функції Рабіна є числа $p$ та $q$. Тому, коли вони відомі, то обернення функції \eqref{eq:7.1} (отримання значення оберненої до $y$ функції $y ^{\frac{1}{2}} \bmod n$) обчислюється за
\begin{equation} \label{eq:7.2} 
L_1 = O(\log{n})
\end{equation}
операцій піднесення до квадрату та множення за модулем $n$.
\item У випадку, коли розклад $n = p \cdot q$ невідомий, то складність обернення становить

\begin{equation} \label{eq:7.3}
L_2 = O(\exp{(\ln^{\frac{1}{2}}n \cdot (\ln{\ln{n}})^{\frac{1}{2}})}),
\end{equation}
що є доведено, або
\begin{equation} \label{eq:7.4}
L_3 = O(\exp{(\ln^{\frac{1}{3}}n \cdot (\ln{\ln{n}})^{\frac{2}{3}})}),
\end{equation}
що припускається.

\end{enumerate}

\begin{algorithm}{Обернення функції Рабіна при відомому розкладі $n = pq$}
\begin{enumerate}
\item Обчислення $y ^{\frac{1}{2}} \bmod p$, де p - просте число.
Існує три можливі випадки:
\begin{itemize}
        \item $p=4m+3$,
        \item $p=8m+5$,
        \item $p=8m+1$, де $m \in \mathbb{Z}$.
\end{itemize}

Нагадаємо критерій Ойлера:
нехай $p>2$ , \textsl{p}--просте, $(y,p)=1$, тоді
\begin{equation}
y^{\frac{p-1}{2}} \equiv \begin{cases}
          \text{1 mod p}, &\text{якщо y квадратичний лишок за модулем p} \\
         \text{-1 mod p}, &\text{якщо y нелишок за модулем p}
        \end{cases}
\end{equation}

Маємо \textsl{y}--квадратичний лишок, і зараз розглянемо ці випадки більш детально. \par

\textit{Випадок} $p=4m+3$:  \\
\[
y^{\frac{p-1}{2}}=1 \bmod p \Rightarrow y^{2m+1}=1 \bmod p \Rightarrow y^{2m+2}=y \bmod p \Rightarrow  y ^{\frac{1}{2}} = \pm y^{m+1} \bmod p.
\]

\textit{Випадок} $p=8m+5$: \\
\[
y^{\frac{p-1}{2}}=1 \bmod p \Rightarrow y^{4m+2}=1 \bmod p \Rightarrow y^{2m+1}= \pm 1 \bmod p .
\]
\begin{remark}
2 -- квадратичний нелишок за модулем $p=8m+5$. \\
Покажемо це:  $\Legendre{2}{p} = (-1)^{\frac{p^2-1}{8}}=-1$. Тоді $2^{\frac{p-1}{2}}=2^{4m+2}=-1\bmod p$
\end{remark}
\begin{align*}
y^{2m+1}&=1 \bmod p & y^{2m+1} &=-1 \bmod p \\
y^{2m+2}&=y \bmod p & y^{2m+1}\cdot 2^{4m+2} &=1 \bmod p \\
y^{\frac{1}{2}} &= \pm y^{m+1} \bmod p & y^{2m+2}\cdot 2^{4m+2} &= y \bmod p \\
 && y^{\frac{1}{2}}&=\pm y^{m+1} 2^{2m+1} \bmod p 
\end{align*}

\textit{Випадок} $p=8m+1$: \\
\[
y^{\frac{p-1}{2}}=1 \bmod p \Rightarrow y^{4m}=1 \bmod p \:, m=2^sd
\]
\[
 y^{2m}= \pm 1 \bmod p. 
\]
Далі розглядається 2 випадки як і раніше, усе робиться аналогічно. У випадку $-1$ у правій частині, шукаємо таке \textsl{b}, яке являється квадратичним нелишком та використовуємо його декілька разів в порівняннях доки не откримаємо $+1$ у правій частині. 

\begin{align*}
y^{2m}&=1 \bmod p & y^{2m} &= -1 \bmod p \\
y^{m}&=1 \bmod p   & y^{2m}\cdot b^{4m} &=1 \bmod p \\
&\vdots & y^{m}\cdot b^{2m} &= 1 \bmod p \\
y^d &=1 \bmod p         & &\vdots \\
y^{\frac{1}{2}} &= y^{\frac{d+1}{2}} \bmod p &&
\end{align*}

\begin{remark}
Якщо $p=4m+3$, то $p$ називається простим числом Блюма. \\
Якщо $n = p \cdot q, p=4m+3, q=4k+3$, то число $n$ називається числом Блюма.
\end{remark}

\item Обчислення $y ^{\frac{1}{2}} \bmod q$, де q -- просте число -- аналогічно першому пункту.

\item Обчислення $y ^{\frac{1}{2}} \bmod n$, де $n = p \cdot q, p\neq q$ (власне, обернення функції Рабіна)

Оскільки $p\neq q$, то $(p, q) = 1$. Тоді маємо систему:
\begin{equation*}
\begin{cases}
y^{\frac{1}{2}}=\beta_p \bmod p,\\
y^{\frac{1}{2}}=\beta_q \bmod q.
\end{cases}
\end{equation*}
За китайською теоремою про лишки: 
\[
y^{\frac{1}{2}} \bmod n = [\pm\beta_p \cdot q \cdot (q^{-1}\bmod p) \pm \beta_q \cdot p \cdot(p^{-1} \bmod q)]\bmod n \:.
\]

\end{enumerate}
\end{algorithm}


\begin{algorithm}{Факторизація n = pq при відомих чотирьох коренях функції Рабіна}
Маємо функцію Рабіна $y = x ^{2} \bmod n$, де $n = p \cdot q,\; p \neq q$. Нехай відомі її корені $x_1, x_2, x_3, x_4$. Знайдемо серед них такі два корені $x_i$ та $x_j$, що $x_i \neq \pm x_j$. Маємо систему:
\begin{equation*}
\begin{cases}
        x^2_i = y \bmod n,\\
        x^2_j = y \bmod n
\end{cases}.
\end{equation*}
Віднявши від верхньої частини нижню, маємо: $(x_i-x_j)(x_i+x_j) = 0 \bmod n$. Але $(x_i-x_j) \notdivisible n$ та $(x_i+x_j) \notdivisible n$. Отже, $(x_i+x_j, n) = p$ або $(x_i+x_j, n) = q$.
За алгоритмом Евкліда знаходимо $p$ або $q$ та, відповідно, $q = \frac{n}{p}$ або $p = \frac{n}{q}$.
\end{algorithm}
\begin{claim}
Задача добування квадратного кореня в функції Рабіна $y = x ^{2} \bmod n$, де $n = p \cdot q,\; p \neq q$, поліноміально еквівалентна задачі факторизації $n = pq$.
\end{claim}