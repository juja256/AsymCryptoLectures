\section{Складність алгоритмів}
\begin{flushright}
\emph{(Автор: Антон Вихло. Не редагувалось.)}
\par \emph{(Версія від 18 січня 2017 р.)}
\end{flushright}

\emph{Алгоритм} -- загальна послідовна процедура, яка виконує покрокове рішення певної задачі.

\emph{Вхід алгоритму} -- деяка скінченна множина, представлена у визначеній системі кодування даних.

Розмір даних можна оцінювати як число символів вхідних даних у певній системі кодування.

Після зупинки алгоритму, результат роботи представляється набором символів в певній системі кодування.

\emph{Часова складність алгоритму} визначається числом кроків до зупинки(або часу, наприклад, роботи ЕОМ).

\emph{Алгоритм} -- дискретна процедура, в якій проміжні та кінцеві результати змінюються покроково.

\emph{Ємнісна складність алгоритму} оцінюється максимальною кількістю символів, що оброблюються на кожному кроці. При реалізаціїї алгоритмів, наприклад, на ЕОМ, максимальний об'єм необхідної пам'яті.
\begin{center} Формальні теорії алгоритмів \end{center}
\begin{enumerate} 
\item Детерміновані машини Т'юрінга(ДМТ), недетерміновані машини Т'юрінга(НДМТ).
\item Рекурсивні функції.
\item Нормальні алгоритми Маркова.
\item Теория Поста.
\end{enumerate} 

\begin{center} Узагальнений тезис Черча \end{center}


Будь-який интуітивно зрозумілий алгоритм може бути сформульований в будь-якій із теорій 1-4.

Позначення:

 $A_n$ - алгоритм.

$X$ - область визначення алгоритму(вхідні дані, що належать $X$).

$ x\in X, x$ - індивідуальний вхід(дані).

$||x||$ - розмір входу.

$T(n,x), x \in X, n=||x||$ - часова складність алгоритму  $A_n$ .

$S(n,x), x \in X, n=||x||$ - ємнісна складність  алгоритму  $A_n$ .

\emph{Середньою складністю} називається 

$T_{\text{сер.}}(n) = \sum\limits_{x \in X, ||x||=n}^{} p(x)\cdot T(n,x) , x \in X, n=||x||,  p(x)$ - деякий розподіл імовірностей на множині $X$.

\emph{Складністю в найгіршому випадку} називається 

$T_{max}(n) = \smash{\displaystyle\max_{x \in X, ||x||=n}} T(n,x)$.

\begin{center} Асимптотичні оцінки складності \end{center}

\emph{Складність для майже всіх входів} називаєтьсятака оцінка $T_{\text{м.в.}}(n)$, що 

$\forall  \epsilon  \ge 0$ $\exists  n_0$  така, що відносна частина тих входів  $ x \in X, n=||x||$, для яких  $T(n)\hm \ge  T_{\text{м.в.}}(n) \to  0$, при  $n \to \infty $.

Аналогічно визначається $S_{\text{сер.}}(n), S_{max}(n), S_{\text{м.в.}}(n).$

 \begin{mydef} Пролономіальною часовою складністю називається така $T(n)$, що $\exists p(x)$~-- поліном, і $T(n) = \mathcal O (p(n)), n \to \infty$, де пiд $T(n)$ розуміється будь-яка складність  $T_{\text{сер.}}(n), T_{max}(n), T_{\text{м.в.}}(n).$\end{mydef}

 \begin{mydef}Експоненційною часовою складністю називається $T(n) \asymp a^{cn}, a = const \hm \ge 1, c = const \ge 0, n \to \infty$. \end{mydef}

Інше означення $T(n) = \theta (a^{cn}).$

 $T(n) \asymp a^{cn} \Longleftrightarrow  T(n) = \mathcal O (a^{cn})$ і $ a^{cn} = \mathcal O(T(n))$

Інакше $\exists 0 \le c_1 \le c_2 \le \infty$ такі, що $c_1a^{cn}\leq T(n) \leq c_2a^{cn}$

Зазвичай $ 0 \le c \le 1,$ $a = 2; e; 10.$ 

 \begin{mydef} Субекспоненційною оцінкою складності (часової) називається така оцінка  $T(n) \asymp \exp (n^\gamma (\ln n)^{1-\gamma})$, $ 0\leq \gamma \leq 1, n \to \infty$. \end{mydef}

При $\gamma \to 0$ субекспоненційна оцінка наближається до поліноміальної.

При $\gamma \to 1$ субекспоненційна оцінка наближається до експоненціальної.

Аналогічно визначаються поліноміальні, експоненційна та субекспоненційні оцінки $S(n)$.

 \begin{mydef}Класом $P$ називаэться клас задач для яких $\exists$ поліноміальний часовий алгоритм (в найгіршому випадку) на ДМТ рішення задачі. \end{mydef}

 \begin{mydef}Класом $NP$ називається клас задач для яких $\exists$ поліноміальний часовий алгоритм (в найгіршому випадку) на НДМТ рішення задачі. \end{mydef}

Відомо: $P \subseteq NP.$

Не доведено: $ P = NP$ або $NP \backslash P \ne \emptyset$.
