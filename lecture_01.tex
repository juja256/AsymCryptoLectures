\section{Односторонні функції. Схема розподілення ключів відкритими каналами.}
\begin{flushright}
\emph{(Автор: Катя Астаф'єва. Частково редагувалось.)}
\par \emph{(Версія від 18 січня 2017 р.)}
\end{flushright}

У минулому семестрі ми познайомилися з симетричною криптографією.  Цього року – з асиметричною. 

Математичні поняття, зв’язані з асиметричною криптографією були настільки нові, що виник, дійсно,  новий напрямок криптографії. Можливості розширилися дуже сильно. А що викликало появу принципово нової криптографії?

Виникло 2 основні проблеми, з якими симетричній криптографії все важче і важче було справлятися.

Проблеми симетричної криптографії у 70-х роках 20-го століття:

1.)Розповсюдження(передача) таємного ключа.

У симетричній криптографії відправник і отримувач повинні мати один і той самий таємний ключ (який знають лише вони і ніхто інший), котрий передається по закритому каналу(часто – перевозили вручну). Проте, раніше криптографією займалися лише військово – дипломатичні відомства. На сьогодні, захистом інформації займаються не лише вони. (Буквально будь-яка галузь науки, промисловості, усі сфери). Тому, експоненційно росте кількість користувачів. Розглянемо простий приклад.

\begin{example}
10 тис. користувачів хочуть спілкуватися незалежно. Скільки їм потрібно ключів?

$$C_{10000}^{2} =\frac{10^4(10^4 -1)}{2} = 5(10^7) = 50000000$$

Висновок: Не вистачить населення Києва.

Питання: Хто буде їх перевозити?
\end{example}

Друга проблема, що розв’язувалася ще важче симетричними можливостями – аутентифікація.

2.) Аутентифікація – підтвердження достовірності автора повідомлення.

В асиметричній криптографії основне поняття, на якому базується стійкість – поняття односторонньої функції. 

У 1976 році зявляється стаття Діффі і Хелмана «Новий напрямок у криптографії». Стаття повністю виправдала назву. У ній введено поняття односторонньої функції, запропонована конкретна одностороння функція , схема розподілення ключів відкритими каналами, схема криптографії з відкритими ключами. Таким чином вони розв’язали першу проблему.

Виявляється, таємні ключі можна не передавати закритими каналами. Це виглядало абсурдно. Але як ми побачимо згодом – таємні ключі не розповсюджуються відкритими каналами, а передається певна інформація, що дозволяє ці ключі побудувати так, що не диллячись на передачу відкритими каналами вони залишаються у таємниці. Для того щоб вирішити таке питання Діффі і Хелман запропонували нове поняття - «одностороння функція». Розглянемо означення, що було ними запропоноване.

\begin{mydef}
Односторонньою функцією називається відображення на скінченних множинах, а саме f(x) : X->Y, таке, що :

1) $\forall x \notin \mathbb X\colon$
існує поліноміальний алгоритм обчислення у = f(x)

2) Для майже $\forall y\notin \mathbb Y \colon$
не існує поліноміального (ефективного) алгоритму обчислення оберненої 
функції $f^{-1}(y)$.

X, Y – великі за потужністю

f(x) – не обовязково бієкція.
\end{mydef}

Поняття оберненої функції ми розуміємо в узагальненому сенсі(тобто, всі праобрази). Коли говорять, що неможливо  ефективно обчислити обернену функцію – мають на увазі, що неможливо ефективно обчислити навіть 1 із таких прообразів.


Чому не для всіх y, а майже для всіх? Якщо казати про практику застосування, то функції завжди мають слабкі(нерухомі) точки. Тобто, ми шифруємо, а текст лишається таким самим. Але, це не головне у слові «майже». В одну сторону функція обчислюється дуже швидко, у іншу – ні. Якщо так – побудуємо  табличку. Візьмемо x і обчислимо y. Якщо хтось захоче обернути функцію для конкретного y – він візьме і подивиться у таблиці. 

Видно, що потужність Х повинна бути досить великою, задля того, щоб таблиця, яку можливо побудувати містила у собі мізерну часту усіх можливих точок.

Визначення – це звичайно добре, але потрібно було запропонувати конкретну функцію.

\textbf{Одностороння функція дискретного піднесення до степеня(функція Діффі-Хелмана)}

$y = \alpha^x \bmod p$       (1)

Область визначення: X = {1,2,...,p-1}

Область значення: Y = {1,2,...,p-1} , т.е X = Y

На множині X функція (1) - бієкція, тобто взаємно - обнозначне відображення.

Як знаходити такі великі прості числа? На практиці p  має 1024, або 2048 біт. Існують алгоритми, наприклад Міллера- Рабіна, Соловея - Штрасена.

Щодо вибору примітивного елементу використаємо:

$\alpha$ - примітивний $\Leftrightarrow \alpha^\frac{p-1}{p_{i}}\equiv 1, \forall i = 1..r$, 
де $p-1 = \sum_{i=1}^r p_i^{k_i}$, $p_{i}$ -- прості, не рівні.

Оцінки складності

Оцінимо:

1. Складність обчислення f(x) $\forall x \in \mathbb X $ запишемо x = $\sum_{i=0}^{r-1} x_i2^i $, r - число двійкових розрядів, тоді


f(x) = $\alpha^{\sum_{i=0}^{r-1} x_i2^i} = \alpha^{x_0}(\alpha^{2})^{x_1}...(\alpha^{2^{r-1}})^{x_{r-1}} mod p$

Тому, оцінимо скільки тут буде операцій.

Складність обчислення $L_1 \leq 2(r-1) \leq 2[log p] \approx 2\log_2 p$

Оцінимо $r$: $r = [\log_2(x+1)]$

2. Оцінки узагальнення(тобто, обчислення дискретного логарифму)

$L_2 = O( \sqrt{p})$ - не найшвидший, але можливий для реалізації.

$L_3 = exp{(c_0 + o(1)ln^{1/3}p(lnlnp)^{2/3}}$, де $c_0 \approx 1,923$- рахують найшвидшим, але це не доведено.


\begin{example}
$L_1  \leq 2log10^{300}\approx 600 * 3,3 \approx 2000$ - швидко

$L_2 = O(\sqrt{10^{300}}) = O(10^{150})$ - число операцій немислимо.

$L_3 \approx 10^{30}$(значно менше, але зробити неможливо)
\end{example}

\textbf{Схема Діффі-Хелмана розподілу ключів по відкритим каналам}

2 користувача А і В вирішили побудувати секретний ключ, використовуючи відкритий канал.

Ціль: побудувати секретний ключ.

Кроки алгоритму : 

1.) А і В обирають, просте p і примітивний елемент $\alpha$.

2.) А генерує випадкове $x_A \in \mathbb X/ 1, p-1$, тобто $x_A = {2,3...p-2}$, $x_A$  – секрет А. Обчислює $\alpha ^{x_A} mod p = y_A$ (передає В по відкритому каналу).

3.) В генерує випадкове $x_B \in \mathbb X/ 1, p-1$, тобто $x_B = {2,3...p-2}$, $x_B$  – секрет B. Обчислює $\alpha ^{x_B} mod p  = y_B$ (надсилає А по відкритому каналу)

4.) А бере число В $(y_B)$ , обчислює $y_B^{x_A} mod p = Z_A = k$

5.)  В – аналогічно, тобто $ y_A^{x_B} mod p = Z_B = k$
Числа будуть рівними 


$z_A = y_B^{x_A} mod p = (\alpha ^{x_B})^{x_A} mod p = \alpha ^{x_Ax_B} mod p = k$


$z_B = y_A^{x_B} mod p = (\alpha ^{x_A})^{x_B} mod p = \alpha ^{x_Ax_B} mod p = k$

k - спільний секрет.

Подивимося зі сторони криптоаналітика.

Криптоаналітик знає : p, $\alpha, y_A, y_B$(перехоплені) 

Чи зможе він обчислити к?

Варіант обчислення к: із $\alpha ^{x_A} mod p = y_A $ знаходимо $x_A$ і обчислюємо $y_B^{x_A} mod p = k$.

Не зможе знайти $x_A$ (потрібно обернути односторонню функцію).

Якого розміру буде секрет к?

К має 1024 біта з вірогідністю приблизно 50 відсотків(трохи більше), 1023 з вірогідністю трохи меншою ніж 50 відсотків.

Таким чином можна не використовувати закритий канал.

Покажемо слабкість даного алгоритму (задача аутентифікації не розв’язана у такому виді).

\textbf{Атака на схему Діффі-Хелмана}

А обчислює $\alpha ^{x_A} mod p = y_A $ , надсилає B 

В обчислює $\alpha ^{x_B} mod p = y_B $ , надсилає A

Але є криптоаналітик Е, котрий перехоплює $y_A$, $y_B$ обирає своє $x_E$ , обчислює $\alpha ^{x_E} mod p = y_E $ , відправляє А і В.

У результаті А і Е побудували спільний ключ, В і Е також побудують спільний ключ. 

А і В починають переписуватися але все перехоплює Е, читає, перешифровує на спільних ключах і надсилає.

Е залишається <<прозорим>>
