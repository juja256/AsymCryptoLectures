\section{Односторонні функції. Схема розподілення ключів відкритими каналами.}
\begin{flushright}
\emph{(Автор: Катя Астаф'єва. Помітно редагувалось Грубіяном Євгеном.)}
\par \emph{(Версія від 19 січня 2017 р.)}
\end{flushright}

У минулому семестрі ми познайомилися із симетричною криптографією. В цьому семестрі ми вивчатимемо порівняно молодий розділ криптографії - \textit{асиметричні криптосистеми та протоколи}. Нагадаємо що симетрична криптографія передбачає наявність єдиного ключа для зашифрування і розшифрування даних, тобто сторони, які хотіли би вести захищену комунікацію мають один і той самий ключ.

В 60-70х роках XX ст. виникло 2 основні проблеми, з якими симетричній криптографії було все важче і важче справлятися. Це спрвокувало появу нової гілки в криптографії - \textit{асиметричної криптографії}. 

Розлянемо детальніше ці проблеми:
\begin{itemize}
\item Розповсюдження(передача) таємного ключа.

У симетричній криптографії відправник і отримувач повинні мати один і той самий таємний ключ (який знають лише вони і ніхто інший), котрий передається по закритому каналу(часто -- перевозили вручну). Проте, раніше криптографією займалися лише військово-дипломатичні відомства. На сьогодні, захистом інформації займаються не лише вони. (Буквально будь-яка галузь науки, промисловості, усі сфери). Тому, експоненційно росте кількість користувачів. Розглянемо простий приклад.

\begin{example}
10 тис. користувачів хочуть спілкуватися незалежно. Скільки їм потрібно ключів?

\[ C_{10000}^{2} =\frac{10^4(10^4 -1)}{2} = 5(10^7) = 50000000 \]

Висновок: Кількість симетричних ключів росте експоненційно, тому виникає питання яким чином вони будуть доставлятись всім користувачам абсолютно секретно.

\end{example}

\item Друга проблема, що важко розв’язувалася симетричною криптографією – автентифікація користувачів. Автентифікація – підтвердження достовірності автора повідомлення. Маючи сектретний ключ, будь-яка сторона може претендувати на автентичність своїх криптограм, проте перевірити це зможуть тільки інші власники цього ж секретного ключа, розшифрувавши криптограму і ніхто інший.

\end{itemize}

В асиметричній криптографії основне поняття, на якому базується стійкість її протоколів та алгоритмів – \textit{одностороння функція}. 

У 1976 році з’являється стаття Діффі і Хелмана «Новий напрямок у криптографії». Стаття повністю виправдала назву. У ній введено поняття односторонньої функції, запропонована конкретна одностороння функція та схема розподілення ключів відкритими каналами, тобто схема криптографії з відкритими ключами. Таким чином вони розв’язали першу проблему.

Виявляється, таємні ключі можна не передавати закритими каналами. Це виглядало абсурдно. Але як ми побачимо згодом – таємні ключі не розповсюджуються відкритими каналами, а передається певна інформація, що дозволяє ці ключі побудувати так, що не дивлячись на передачу відкритими каналами вони залишаються у таємниці. Для того щоб вирішити таке питання Діффі і Хелман запропонували нове поняття - «одностороння функція» та привели приклад функції, що можливо є односторонньою. Розглянемо означення, що було ними запропоноване.

\begin{mydef}
\textit{Односторонньою функцією} називається відображення на скінченних множинах \( f(x) : X\rightarrow Y \), таке, що :
\begin{enumerate}
<<<<<<< HEAD
\item $\forall x \notin X \colon$ існує поліноміальний алгоритм обчислення $y = f(x)$
=======
\item $\forall x \in X \colon$ існує поліноміальний алгоритм обчислення $у = f(x)$
>>>>>>> a0cfb91d486b7d468ebeecdd29bf9fe485656a68

\item Для майже всіх $y\in Y \colon$ не існує поліноміального (ефективного) алгоритму обчислення оберненої функції $f^{-1}(y)$.
\end{enumerate}
Зауважимо, що X, Y з практичних міркувань стійкості досить потужні множини.\par Взагалі кажучи, $f(x)$ – не обовязково бієкція.
\end{mydef}

Поняття оберненої функції ми розуміємо в узагальненому сенсі(тобто, всі прообрази). Коли говорять, що неможливо ефективно обчислити обернену функцію – мають на увазі, що неможливо ефективно обчислити навіть 1 із таких прообразів.

Чому не для всіх y, а майже для всіх? Якщо казати про практику застосування, то функції завжди мають слабкі(нерухомі) точки. Тобто, ми шифруємо, а текст лишається таким самим. Але, це не головне у слові «майже». В одну сторону функція обчислюється дуже швидко, у іншу – ні. Якщо так – побудуємо  табличку. Візьмемо x і обчислимо y. Якщо хтось захоче обернути функцію для конкретного $y$ – він подивиться відповідний $x$ у таблиці. 

Видно, що потужність $Х$ повинна бути досить великою, задля того, щоб таблиця, яку можливо побудувати містила у собі мізерну часту усіх можливих точок.

\begin{mydef}
\textit{Функція Діффі-Хелмана дискретного піднесення до степеня} має вигляд:
\[ y = \alpha^x \bmod p ,\]
де p -- деяке просте число, а $\alpha$ -- примітивний елемент поля $\GF{p}$
\end{mydef}
Область визначення: $X = \{1,2,...,p-1\}$ \\
Область значень: $Y = \{1,2,...,p-1\}$ , тобто X = Y \\
На множині X функція (1) -- бієкція, тобто взаємно-обнозначне відображення, в силу примітивності $\alpha$

На практиці число p має доволі велику довжину, порядка 1024 чи 2048 двійкових розрядів. Існують алгоритми, наприклад Міллера- Рабіна, Соловея - Штрасена та ін. для пошуку таких великих простих чисел за досить ефективний час.

Щодо вибору примітивного елементу використаємо той факт, що $\alpha$ - примітивний тоді і тільки тоді коли $\alpha^\frac{p-1}{p_{i}}\not\equiv 1 \bmod p, i = \overline{1,\ r}$, 
де $p-1 = \prod\limits_{i=1}^r p_i^{k_i}$, $p_{i}$ - прості, не рівні.

\begin{remark}
На сьогоднішній день не доведено існування бодай одної односторонньої функції. Ця проблема пов'язана з набагато глибшою проблемаю $P \neq NP$, що поки не розв'язана. Тому приведена вище функція називається не односторонньою, а \textit{кандидатом в односторонні}.
\end{remark}
Оцінимо складність обчислення функції Діффі-Хелмана і оберненої до неї.
\begin{itemize}

\item \textbf{Складність обчислення прямої функції}\\
$\forall x \in X $ запишемо x = $\sum_{i=0}^{r-1} x_i2^i $, r - число двійкових розрядів, тоді

\[f(x) = \alpha^{\sum_{i=0}^{r-1} x_i2^i} = \alpha^{x_0}(\alpha^{2})^{x_1}\dots(\alpha^{2^{r-1}})^{x_{r-1}} \bmod p\]

Складність обчислення в такому випадку оцінюється зверху: 
\[L_1 \leq 2(r-1) \leq 2[log\: p] \approx 2\log_2 p\]

\item \textbf{Складність обернення функції Діффі-Хелмана(дискертне логарифмування)}
Даємо деякі оцінки складності алгоритмів дискретного логарифмування без доведень:

$L_2 = O( \sqrt{p})$ - не найшвидший, але можливий для реалізації алгоритм дискретного логарифмування.

$L_3 = exp\{(c_0 + o(1)) ln^{1/3}p(lnlnp)^{2/3}\}$, де $c_0 \approx 1,923$ - вважають найшвидшим.

\end{itemize}

\begin{example}Оцінка складності для числа порядку 1024 біт:

$L_1  \leq 2log10^{300}\approx 600 \cdot 3,3 \approx 2000$ - швидко

$L_2 = O(\sqrt{10^{300}}) = O(10^{150})$ - число операцій дуже велике.

$L_3 \approx 10^{30}$ - значно менше, але все ще недосяжне для сучасної техніки.
\end{example}

Нехай 2 користувача: \textbf{А} і \textbf{В} вирішили побудувати секретний ключ, використовуючи відкритий канал.

\begin{algorithm}[Схема Діффі-Хелмана розподілу ключів по відкритим каналам]\

\begin{enumerate}

\item \textbf{А} і \textbf{В} обирають, просте p і примітивний елемент $\alpha$.

\item \textbf{А} генерує випадкове $x_A \in \{2,3, \dots, p-2\}$, $x_A$  – секрет \textbf{А}.\\
Обчислює $\alpha ^{x_A} \bmod p = y_A$ та передає $y_A$ до \textbf{B} по відкритому каналу.

\item \textbf{B} генерує випадкове $x_B \in \{2,3, \dots, p-2\}$, $x_B$  – секрет \textbf{B}.\\
Обчислює $\alpha ^{x_B} \bmod p = y_B$ та передає $y_B$ до \textbf{A} по відкритому каналу.

\item \textbf{А} обчислює $y_B^{x_A} \bmod p = z_A = k$

\item \textbf{B} обчислює $y_A^{x_B} \bmod p = z_B = k$
\end{enumerate}
k - спільний секрет.

\end{algorithm}
Перевіримо що $z_A=z_B$
\[ z_A = y_B^{x_A} \bmod p = (\alpha ^{x_B})^{x_A} \bmod p = \alpha ^{x_Ax_B} \bmod p = k \]
\[ z_B = y_A^{x_B} \bmod p = (\alpha ^{x_A})^{x_B} \bmod p = \alpha ^{x_Ax_B} \bmod p = k \]

Криптоаналітик \textbf{E} знає: $p,\ \alpha,\ y_A,\ y_B$(перехоплені). Чи зможе він обчислити $k$?

Варіант обчислення $k$: із $\alpha ^{x_A} mod p = y_A $ знаходимо $x_A$ і обчислюємо $y_B^{x_A} mod p = k$.
Але спроби криптоаналітика знайти $x_A$ будуть марними, оскільки це еквівалентно знаходженню дискретного логарифму, що є складною задачею.

Спільний секрет $k$ має довжину 1024 біта з імовірністю трохи більшою за 50\%, 1023 з імовірністю трохи меншою ніж 50\%.
Таким чином можна не використовувати закритий канал.

Покажемо слабкість даного алгоритму до атак типу людина посередині(\textit{Man in the middle attack}).

\begin{algorithm}[Атака на протокол Діффі-Хелмана типу людина посередині]\
\begin{enumerate}
\item \textbf{А} обчислює $\alpha ^{x_A} \bmod p = y_A $, надсилає \textbf{B} 

\item \textbf{В} обчислює $\alpha ^{x_B} \bmod p = y_B $ , надсилає \textbf{A}

\item Криптоаналітик \textbf{Е}, котрий перехоплює $y_A$, $y_B$ обирає своє $x_E$ , обчислює $\alpha ^{x_E} mod p = y_E $ та відправляє його \textbf{А} і \textbf{В}.

У результаті \textbf{А} і \textbf{Е} побудували спільний ключ, \textbf{В} і \textbf{Е} також побудували спільний ключ. 
\textbf{А} і \textbf{В} починають комунікувати, але все перехоплює \textbf{Е}, розшифровує, можливо модифікує, перешифровує на спільних ключах і надсилає другому абонентові, при цьому присутність криптоаналітика ніяк не помітна для абонентів \textbf{A} та \textbf{B}.
\end{enumerate}

\end{algorithm}
Таким чином схема Діффі-Хелмана не вирішує задачі автентифікації.
