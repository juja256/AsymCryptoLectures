\documentclass[12pt]{report}
\usepackage[pdftex]{graphicx}
% Comment the following line to NOT allow the usage of umlauts
\usepackage[utf8]{inputenc}
\usepackage[russian, ukrainian]{babel}

\usepackage{amsthm,amsfonts,amsmath,amssymb,amscd}  % Математические дополнения от AMS
\usepackage{geometry}                               % Для последующего задания полей
\usepackage{indentfirst}                            % Красная строка
\usepackage[singlelinecheck=off,center]{caption}    % Многострочные подписи
\usepackage{soul}                                   % Поддержка переносоустойчивых подчёркиваний и зачёркиваний
\usepackage{icomma}                                 % Запятая в десятичных дробях
\usepackage{tocloft}
\usepackage{setspace}
\usepackage{fancyhdr}
\usepackage{euscript}                               % Красивый шрифт \EuScript
%%% Цвета %%%
\usepackage[usenames]{color}
\usepackage{color}
\usepackage{colortbl}

\geometry{a4paper,top=2cm,bottom=2cm,left=2.5cm,right=1cm}
\linespread{1.0}                    % Одинарный интервал
\sloppy                             % Избавляемся от переполнений
\clubpenalty=10000                  % Запрещаем разрыв страницы после первой строки абзаца
\widowpenalty=10000                 % Запрещаем разрыв страницы после последней строки абзаца

%%% Шаблон %%%
\newcommand{\todo}[1]{\textcolor{red}{#1}}
% Команда \todo{Текст} вставляет текст красного цвета

% Используем дефис для ненумерованных списков (ГОСТ 2.105-95, 4.1.7)
\renewcommand{\labelitemi}{\normalfont\bfseries{--}} 

%%% Колонтитулы %%%
% Порядковый номер страницы печатают на середине верхнего поля страницы (ГОСТ Р 7.0.11-2011, 5.3.8)
\makeatletter
\let\ps@plain\ps@fancy              % Подчиняем первые страницы каждой главы общим правилам
\makeatother
\pagestyle{fancy}                   % Меняем стиль оформления страниц
\fancyhf{}                          % Очищаем текущие значения
\fancyhead[R]{\thepage}             % Печатаем номер страницы в правом краю верхнего поля
\renewcommand{\headrulewidth}{0pt}  % Убираем разделительную линию

%переопределение символа пустого множества
\let\oldemptyset\emptyset
\let\emptyset\varnothing

%определение команд для частых символов
\newcommand*{\binsp}[1]{\ensuremath \left\{0, 1\right\}^{#1}}       % {0, 1}^m
\newcommand*{\xor}{\ensuremath \oplus}                              % \xor = (+)
\newcommand*{\GF}[1]{\ensuremath \mathbb F_{#1}}                    % F_n
\newcommand*{\GFgroup}[1]{\ensuremath \mathbb F^{*}_{#1}}           % F^*_n
\newcommand*{\Zring}[1]{\ensuremath \mathbb Z_{#1}}                 % Z_n
\newcommand*{\Zgroup}[1]{\ensuremath \mathbb Z^{*}_{#1}}            % Z^*_n
\newcommand*{\Jset}[1]{\ensuremath \mathbb J_{#1}}                  % J_n
\newcommand*{\Qset}[1]{\ensuremath \mathbb Q_{#1}}                  % Q_n
\newcommand*{\PQset}[1]{\ensuremath \widetilde{\mathbb Q}_{#1}}     % Q~_n
\newcommand*{\Legendre}[2]{\ensuremath \left(\frac{#1}{#2}\right)}  % символ Лежандра/Якоби

\DeclareMathOperator{\ord}{ord}         % фунция ord

%определение команды для переноса знаков операции при разрыве строки
\newcommand*{\hm}[1]{#1\nobreak\discretionary{} {\hbox{$\mathsurround=0pt #1$}}{}}
%использовать как: $a \hm+ b \hm+ c$; при переносе на новую строку знак будет продублирован

%определение окружений типа "теорема"
% Теоремы, леммы, утверждения, следствия: курсивным текстом, нумерация по главе (вида 1.2)
\theoremstyle{plain}
\newtheorem{theorem}{Теорема}[chapter]
\newtheorem{claim}{Твердження}[chapter]
\newtheorem{lemma}{Лема}[chapter]
\newtheorem{corollary}{Наслідок}[chapter]

% Определения, алгоритмы, замечания и задачи: прямым текстом, нумерация по главе
% (замечания не нумеруются)
% Доказательства (окружение "proof") относятся сюда же.
\theoremstyle{definition}
\newtheorem{mydef}{Визначення}[chapter]
\newtheorem{algorithm}{Алгоритм}[chapter]
\newtheorem{problem}{Задача}[chapter]
\theoremstyle{remark}
\newtheorem*{remark}{\textbf{Зауваження}}
\renewcommand{\proofname}{\textbf{Доведення}}

%команда для английских названий
\newcommand{\engl}[1]{(англ. \emph{#1})}
% команда \engl{text} вставляет в текст фразу "(англ. __text__)".
% Сделано для дублирования терминов на английском


% Start the document
\begin{document}

\chapter{Назви розділів сенсей вставить сам}

%\section{Що таке електронні вибори?}
%
%Пока кандидат на кат

\section{Приклади}

Наразі вже декілька розвинутих країн практикують електронні вибори. Тобто, держава дає можливість
людині проголосувати за свого кандидата не виходячи з дому.

Можна розглянути, як це реалізовано на прикладі такої країни як Естонія.

Кожному громадянину Естонії, коли йому виповнилося 15 років, видається так звана ID-картка, що виконує
функції паспорта в нашому розумінні. Однак, на відміну від звичайного паспорту в нії можна зберігати
більше інформації про людину. Її навіть використовують як закордонний паспорт в деяких країнах.

\includegraphics{idcard}

Вже у 2005 році вперше була випробована на реальному голосуванні система електронних виборів.
Вона була визнана успішною. 
Естонія стала першою країною світу, де стало можливим проголосувати через Інтернет. 

\section{Вимоги до електронного голосування}
Задля того, щоб побудувати систему електронного голосування, спочатку треба визначитися з вимогами, які
ця система повинна задовольняти:


\begin{enumerate}
    \item голосувати може лише зареєстрована людина;
    \item кожен може проголосувати лише один раз;
    \item усі голоса враховуються анонімно: зберігається таємниця виборів;
    \item проголосувавши, ми не маємо змоги змінити своє рішення;
    \item неможливість підробки результатів голосування;
    \item повинна бути можливість відкрито перевірити результати голосування;
    \item ми повинні мати змогу взнати, скільки людей проголосувало.
\end{enumerate}

Як ми бачимо, виставленні вимоги до системи, десь співпадають, а десь перевищують можливості звичайного
голосування. Проте давайте розглянемо, як усю цю купу вимог реалізувати на практиці.

\section{Схема електронного голосування}

Зазвичай організацією виборів займається якась державна структура.
В електронному голосуванні також маємо структуру, яка відповідає за організацію 
виборів. Назвемо її виборчим центром $\Sigma$.

В нашого виборчого центру, як і в кожного чемного виборчого центру існує довірена
особа $D$, яка буде виконувати функції обробки голосів виборців.

Самих виборців позначимо $A_i, i=\overline{1,N}$. Звісно, що в кожного виборця є
якась своя унікальна комбінація параметрів $D_i$, яка його ідентифікує. 
Також, у кожного виборця є своя RSA пара відкритого ключа $(n_i,e_i)$ і 
секретного ключа $d_i, i=\overline{1,N}$.

Перед початком самих виборів ми повинні скласти список усіх учасників процесу.
Зазвичай, цим займається виборчий центр. Виходячи з тієї інформації, що в нас є
логічно зберігати список виборців у вигляді:

\[ T = \left \{ <D_i, (n_i,e_i), i = \overline{1,N} \right\} \]

У довіреного лиця також є своя RSA із відкритим ключем $(m,e)$ і таємним ключем $d$.

\section{Процедура голосування виборців}

Нехай в нас існує $r$ можливих варіантів, за кого голосувати. $r-1$ кандидат і один варіант "Проти всіх".

Розглянемо приклад, коли в нас є наступні кандитати:

\begin{enumerate}
    \item Фідель
    \item Мао
    \item Наполеон
    \item Нельсон
\end{enumerate}

Спочатку, виборчий центр обирає кодування для кандидатів. Візьмемо варіант, коли $i-ий$ кандидат
представлен $i-им$ простим числом. Тобто для нашого варіанту маємо:

\begin{itemize}
    \item Фідель 2
    \item Мао 3
    \item Наполеон 5
    \item Нельсон 7
    \item Проти всіх 1
\end{itemize}

І ось настає славетна година, коли виборець $A_i$ йде на вибори -
до компьютера голосувати за свого кандидата $b_i$. 

Спочатку він, як і кожен чемний виборець, сгенерує своє просте число $q_i$, котре більше за 
усі числа-коди кандидатів. (У нашому прикладі $q_i>7$)

Обчислює $t_i=b_i*q_i$ за умови, що $b_i*q_i<m$, де $m$ - це частина відкритого ключа для $D$.

Після цього він зашифровує свій власний вибір $t_i$ відкритим ключем довіреного лиця $D$
\[ C_i = (t_i)^e mod\ m \].

Коли задача конфіденційності вирішена, потрібно ще вирішити задачу цілісності, тому ми підписуємо 
нашу "бюлетень" цифровим підписом:

\[ S_i = (h(D_i || C_i))^{d_i} mod \ m \] 
де $i = \overline{1,N}$, а h(.) - геш-функція

І тільки після цього ми передаємо свій зашифрований бюлетень виборчому центру:

\[ M_i = (D_i,\ C_i,\ S_i) \rightarrow \Sigma\]  

Після того, як усі охочі виборці зроблять свій вибір, вибори закриваються, і ми переходимо до етапу 
підрахунку голосів.

\section{Підрахунок голосів $\Sigma$ -ою}

Спочатку, довірча особа $D$ перевіряє цифровий підпис кожного виборця $A_i, i=\overline{1,N'}$:
\begin{enumerate}
    \item Спочатку йде перевірка, чи була особа $D_i$ зареєстрована для участі у виборах.
    \item Потім перевіряємо валідність цифрового підпису: 
    \[h(D_i||C_i)^{e_i}\ mod\ n_i\ ?=?\ h(D_i, C_i) \]
    
    \item Далі, якщо усі перевірки до цього пройдені, то $D$ розшифровує повідомлення:
    \[ {t_i}^d\ mod\ n = b_i\ *\ q_i\]
    та помічаємо $A_i$, як проголосувавшого.

    \item Останнім кроком $D$ вираховує значення:
    \[ Q = \prod_{i=1}^N \ b_i\ *\ q_i\]

    та відправляє його до виборчого центру $\Sigma$.
\end{enumerate}

Виборчий центр розкладує $Q$ на множники 
\[ Q = 2^{r_1}\ *\ 3^{r_2}\ *\ 5^{r_3}\ *\ 7^{r_4}\ *\ R\]
де $R = \prod_{i=1}^{N'}\ q_i$

І саме з розкладу визначає розподіл голосів по кандидатах.

Після підрахунку результати публікуються у формі:
\[ \{\ \{<D_i,C_i>, i=\overline{1,N'}\}\ ,\ r_1,\ r_2, ...\ ,\ r_k,\ N-\sum_{i=1}^{N'}r_i ,\ R,\ N-N' \} \]

де $N-\sum_{i=1}^{N'}r_i$ - це кількість голосів відданих варіанту "Проти всіх";

$N-N'$ - це кулькість не голосувавших виборців.

\section{Перевірка результатів голосування}

\begin{enumerate}
    \item Кожен виборець має змогу впевнетися, що його голос був зарахований по таблиці результатів.
    Тобто, воно буде присутнє в опублікованих даних виборів.

    Якщо ж виборець не голосував, то він не знайде себе серед опублікованих зашифрованих повідомлень.
    
    \item Таємниця голосування забезпечується для усіх, окрім довіреного лиця $D$.

    \item Як ми можемо бачити із схеми голосування, проголосувати за когось іншого неможливо. Підробити
    голос також.

    \item Перевірку правильності підрахунку голосів може перевірити будь-яка людина.

    Перш за все, ми перевіряємо те, щоб $R$ не повинно ділитися на 2, 3, 5, 7 

    З результатів виборів легко відновлюється число $Q$. Задля того, щоб перевірити правильність
    результатів, усе, що нам потрібно - це впевнетися, що має місце такая рівність:
    \[ Q^e \ mod \ m = (\prod_{i=1}^{N'} C_i) \ mod \ m\]

    Дійсно, якщо усе вірно, то:
    \[ Q^e \ mod \ m = (\prod_{i=1}^{N'} 2^{r_1} * 3^{r_2} * .. * 7 ^{r_4} * \prod_{i=1}^N q_i)^e 
    \ mod \ m  = \prod_{i=1}^N (b_i*q_i)^e\ mod\ m = (\prod_{i=1}^{N'} C_i) \ mod \ m\]

\end{enumerate}








\end{document}

